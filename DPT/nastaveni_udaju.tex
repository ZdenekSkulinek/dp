%% Název práce:
%  První parametr je název v originálním jazyce,
%  druhý je překlad v angličtině nebo češtině (pokud je originální jazyk angličtina)
\nazev{Paralelizace Goertzelova algoritmu}{Parallelization of Goertzel algorithm}

%% Jméno a příjmení autora ve tvaru
%  [tituly před jménem]{Křestní}{Příjmení}[tituly za jménem]
\autor[Bc.]{Zdeněk}{Skulínek}

%% Jméno a příjmení vedoucího včetně titulů
%  [tituly před jménem]{Křestní}{Příjmení}[tituly za jménem]
% Pokud vedoucí nemá titul za jménem, smažte celý řetězec '[...]'
\vedouci[Ing.]{Petr}{Sysel}[Ph.D.]

%% Jméno a příjmení oponenta včetně titulů
%  [tituly před jménem]{Křestní}{Příjmení}[tituly za jménem]
% Pokud nemá titul za jménem, smažte celý řetězec '[...]'
% Uplatní se pouze v prezentaci k obhajobě
\oponent[prof.\ Ing.]{Zdeněk}{Smékal}[CSc.]

%% Označení oboru studia
% První parametr je obor v originálním jazyce,
% druhý parametr je překlad v angličtině nebo češtině
\oborstudia{Teleinformatika}{Teleinformatics}

%% Označení ústavu
% První parametr je název ústavu v originálním jazyce,
% druhý parametr je překlad v angličtině nebo češtině
\ustav{Ústav telekomunikací}{Department of Telecommunications} 

%% Rok obhajoby
\rok{Rok}
\datum{7.\,6.\,2017} % Uplatní se pouze v prezentaci k obhajobě

%% Místo obhajoby
% Na titulních stránkách bude automaticky vysázeno VELKÝMI písmeny
\misto{Brno}

%% Abstrakt
\abstrakt{
Technické problémy znemožňují neustále zvyšovat hodinové frekvence procesorů.
Jejich výkon tak v současné době roste díky zvyšování počtu jader.
To s sebou přináší nutnost nových přístupů pro programování takovýchto paralelních systémů.
Tato práce ukazuje, jak využít paralelismus k číslicovému zpracování signálu. Jako příklad zde bude uvedena implementace Geortzelova algoritmu s využitím výpočetního výkonu grafického čipu.
}{
Technical problems make impossible steadily increase processor's clock frequency.
Their power are currently growing due to increasing number of cores.
It brings need for new approaches in programming such parallel systems.
This thesis shows how to use paralelism in digital signal processing.
As an example, it will be presented here
implementation of the Geortzel's algorithm using the processing power of the graphics chip.
}

%% Klíčová slova
\klicovaslova{Goertzelův algoritmus, zpracování signálu, openCL, paralelní výpočet, GPU}%
	{Goertzel's algorithm, signal processing, openCL, parallel computing, GPU}

%% Poděkování
\podekovanitext{Rád bych poděkoval vedoucímu diplomové práce panu Ing.~Petru Syslovi, Ph.D.\ za odborné vedení, konzultace, trpělivost a podnětné návrhy k~práci.}