\chapter{Knihovna openCV}
\label{kap:opencv}

\section{Úvod do openCV}
OpenCV (Open Source Computer Vision Library) dostupná na http://opencv.org \cite{opencv} je open-source BSD-licencovaná knihovna, která zahrnuje několik set algoritmů z počítačového vidění. 
\\
OpenCV má modulární strukturu, jsou v ní zejména moduly uvedené níže:

\begin{itemize}
\item \emph{core} - Základní datové struktury a funkce, zejména pak třída \classname{Mat}.
\item \emph{imgproc} - Filtrace, geometrické transformace a barevné konverze.
\item \emph{video} - Výpočet rozdílu snímků a detekce pohybu.
\item \emph{calib3d} - Základní víceobrazové geometrické transformace.
\item \emph{features2D} - Detektory charakteristických rysů a jejich srovnávače .
\item \emph{objdetect} - Detekce objektů a jejich zařazení do tříd (lidé, oči, obličeje, auta...).
\item \emph{highgui} - Jednoduché uživatelské rozhraní a a interface pro zachytávání obrazu, obrazové a video kodeky. 
\item \emph{gpu} - GPU akcelerované algoritmy ostatních zde uvedených modulů.
\item \emph{ a další...}
\end{itemize}


\section{Základní API}


Knihovna open cv je umístěna ve jmenném prostoru \classname{cv}.
Je to velmi důležité, protože řada jmen funkcí může být stejná jako u jiných knihoven.
Asi nejdůležitější třídou je třída \classname{Mat} reprezentující obecnou matici, se kterou se pracuje téměř stejně jako v programech \srcfilename{matlab}, \srcfilename{octave}. Je tu ale jeden zásadní rozdíl. Třída nemá svoje data, přesněji, několik instancí třídy \classname{Mat} 
může sdílet stejná data. Je to výhodné, protože ve většině případů se tak zamezí kopírování mnoha bytů. Pokud tedy potřebujeme skutečně Matici zkopírovat, použijeme její členskou metodu \functionname{clone}.
Je třeba upřesnit, že \classname{Mat} může být i jen určitá oblast jiné \classname{Mat} a přesto sdílejí společná data. \classname{Mat} definuje možnost mít jako data většinu základních typů, samozřejmě jako \classname{unsigned char}, \classname{signed char}, \classname{usigned short}, \classname{float}, \classname{double}, a  další. Umí samozřejmě několik kanálů. Uvedu krátký příklad:


\begin{Verbatim}
// vytvoří  8MB velkou matici
Mat A(1000, 1000, CV_64F);

// vytvoří jinou hlavičku pro stejná data
Mat B = A;
\end{Verbatim}


\begin{Verbatim}
// vytvoří jinou hlavičku, jako třetí řádek matice A.
// žádná data se nekopírují
Mat C = B.row(3);
\end{Verbatim}


\begin{Verbatim}
// teď vytvoří samostatnou kopii matice B
Mat D = B.clone();
\end{Verbatim}

\begin{Verbatim}
// zkopíruje 5tý řádek matice B do matice C,
// matice C nyní bude mít stejná data jako matice A.
B.row(5).copyTo(C);
\end{Verbatim}


\begin{Verbatim}
// teď A a D sdílí stejná data. Matice B a C ale stále ukazují na data
// staré modifikované matice A
A = D;
\end{Verbatim}


\begin{Verbatim}
// nyní smažeme referenci na data u matice B. Matice C
// ale stále ukazuje na data staré modifikované matice
// A. A to přesto, že ukazuje jen na jeden osamocený řádek těchto dat.
B.release();;
\end{Verbatim}


\begin{Verbatim}
// nakonec, vytvořením úplné kopie matice C zaručíme,
// že původní data budou dealokované, protože
// žádné matice na ně již nebude držet odkaz.
C = C.clone();
\end{Verbatim}
 

Matice se dá snadno vypsat, stačí jí poslat do výstupního streamu.

\begin{Verbatim}
cout<<R;

//nebo

cout << "R (python)  = " << endl << format(R,"python") << endl << endl;
\end{Verbatim}

%\section{První projekt}
%
%Tento projekt, takový \uv{Hello openCV world} jsem si vypůjčil ze
%stránek \url{openCV.org}. Demonstruje, že začít projekt s openCV je skutečně jednoduché. Jsou  v něm použity pouhé 4 funkce: načtení obrázku, jehož jméno
%je dáno jako parametr z~příkazové řádky(\functionname{imread}), potom vytvoření okna (\functionname{namedWindow}) následně zobrazení obrázku v okně (\functionname{imshow}) a nakonec jen čekání na stisk klávesy (\functionname{waitKey}).
%\\
%\srcfilename{Main.cpp}:
%
%\begin{Verbatim}
%#include <stdio.h>
%#include <opencv2/opencv.hpp>
%
%using namespace cv;
%
%int main(int argc, char** argv )
%{
%    if ( argc != 2 )
%    {
%        printf("usage: DisplayImage.out <Image_Path>\n");
%        return -1;
%    }
%
%    Mat image;
%    image = imread( argv[1], 1 );
%
%    if ( !image.data )
%    {
%        printf("No image data \n");
%        return -1;
%    }
%    namedWindow("Display Image", WINDOW_AUTOSIZE );
%    imshow("Display Image", image);
%
%    waitKey(0);
%
%    return 0;
%}
%\end{Verbatim}
%
%Pro překlad je třeba vytvořit si následující soubor:	
%\\
%\srcfilename{CMakeLists.txt}
%
%\begin{Verbatim}
%
%cmake_minimum_required(VERSION 2.8)
%project( test02 )
%find_package( OpenCV REQUIRED )
%add_executable( dist/Debug/GNU-Linux/test02 main.cpp )
%target_link_libraries( dist/Debug/GNU-Linux/test02 \${OpenCV_LIBS} )
%\end{Verbatim}
%\label{sec:compilerules}
%\noindent
%kde test02 je jméno projektu a výstupního spustitelného souboru.
%Pak lze přidat výše uvedený zápis  kamkoliv do projektu.
%Spuštění provádímě jako každého jiného programu, s tím rozdílem, že je třeba nejdříve projekt
%přeložit příkazem \srcfilename{cmake}.
%
%\begin{Verbatim}
%cmake .
%make
%dist/Debug/GNU-Linux/test02 lena.jpg
%\end{Verbatim}